\begin{enumerate}
    \item
          \begin{enumerate}[label=(\alph*), itemsep=3pt]
              \item
          \end{enumerate}

    \item
          \begin{enumerate}[label=(\alph*), itemsep=3pt]
              \item The proof incorrectly defines a subset, stating that there exists an element in the smaller set that's in the larger set rather than that all elements in the smaller set are in the larger set.
              \item True
              \item Suppose $A$, $B$, and $C$ are sets such that $A \subseteq B$ and $B \subseteq C$. Since $A \subseteq B$, every element in $A$ is in $B$. Since $B \subseteq C$, every element in $B$ is in $C$. Therefore, every element in $A$ is in $C$ implying $A \subseteq C$.
          \end{enumerate}

    \item
          \begin{enumerate}[label=(\alph*), itemsep=3pt]
              \item The proof makes the statement that $B \subseteq A \cap B$ based on the givens. However, such a statement is invalid because $B$ can contain elements outside of the set $A \cap B$ because $B$ maybe larger than its intersection with $A$.
              \item False
              \item Suppose $A$, $B$, and $C$ are sets such that $A \cap B \subseteq C$. Lets assume that $B \subseteq C$. Based on the transitive property of subsets, $B \subseteq A \cap B$. However, such a statement is false because $B$ may contain elements outside its intersection with $A$ violating the transitive property of subsets. Therefore, $B \subseteq C$ is false implying the claim is also false.
          \end{enumerate}
\end{enumerate}