\begin{enumerate}
    \item Prove the following via induction. For every $n \in \N$ we have:
          \begin{equation}
              \sum_{i=1}^{n} i^3 = \frac{n^2(n+1)^2}{4}
          \end{equation}
          \begin{enumerate}[label=(\roman*), itemsep=10pt]
              \item \underline{Base Case}:
                    \begin{align*}
                        n     & = 1                        \\
                        (1)^3 & = \frac{(1)^2(1 + 1)^2}{4}
                    \end{align*}
              \item \underline{Inductive Step}: \\
                    Assume that (1) holds true for $n=k$. Consequently, (1) can be proven to hold true for $n = k + 1$. \\
                    Left-hand side:
                    \begin{align*}
                        \sum_{i=1}^{k+1} i^3 & = (k+1)^3 + \sum_{i=1}^{k} i^3                                            \\
                                             & = (k + 1)^3 + \frac{k^2(k+1)^2}{4} \because {\text{induction hypothesis}} \\
                    \end{align*}
                    Right-hand side:
                    \begin{align*}
                        (k + 1)^3 + \frac{k^2(k+1)^2}{4} & = \frac{(k+1)^2(k+2)^2}{4} \\
                        (k+1)^2\left(4(k+1) + k^2\right) & = (k+1)^2(k+2)^2           \\
                        4k + 4 + k^2                     & = k^2 + 4k + 4
                    \end{align*}

              \item Thus, by \textbf{weak induction}, the statement (1) holds for all $n \in \N$.
          \end{enumerate}
    \item Let $n \in \Z$. Consequently, $n+1 \in \Z$. Then, $(n+1)^2 - n^2 = n^2+2n+1-n^2 =2n+1$. Thus, for any $n \in \Z$, every odd integer can be written as the difference of perfect squares since the result is $2n + 1$.
\end{enumerate}