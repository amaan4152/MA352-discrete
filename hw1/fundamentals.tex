\subsection*{Introduction to Sets}
\begin{enumerate}
    \item
    \begin{enumerate}[label=(\alph*), itemsep=16pt]
        \item 
        \begin{enumerate}[label=(\roman*), itemsep=10pt]
            \item $
                \set{ x \in \Z: \abs{2x} < 5 } \;=\; \
                \set{ -2, -1, 0, 1, 2 } 
            $
            \item $
                \set{ X : X \subseteq \set{3,2,a} \text{ and } \abs{X} = 2 } \;=\; \
                \set{ \set{ 3,2 }, \set{ 3,a }, \set{ 2,a } }
            $
            \item $
                \set{ X \subseteq \N : \abs{X} \leq 1 } \;=\; \
                \set{ \emptyset, x },\; x \in \N
            $
        \end{enumerate}

        \item
        \begin{enumerate}[label=(\roman*), itemsep=10pt]
            \item $\set{0, 1, 4, 9, 16, 25, 36}$ \vspace*{0.2cm}
            \begin{enumerate}[label=(\arabic*)]
                \item $\set{ x^2 : x \in \Z_{\geq 0} }$
                \item $\set{ x^2 : x \in \Z \text{ and } x \geq 0 }$
            \end{enumerate}
            \item $\set{3, 4, 5, 6, 7, 8}$ \vspace*{0.2cm}
            \begin{enumerate}[label=(\arabic*)]
                \item $\set{ x \in \N : 3 \leq x \leq 8 }$
                \item $\set{ x \in \Z : 3 \leq \abs{x} \leq 8 }$
            \end{enumerate}
            \item $\set{\ldots, -\pi, \frac{-\pi}{2}, 0, \frac{\pi}{2}, \pi, \frac{3\pi}{2}, 2\pi, \frac{5\pi}{2}, \ldots}$ \vspace*{0.2cm}
            \begin{enumerate}[label=(\arabic*)]
                \item $\set{ \frac{x\pi}{2} : x \in \Z }$
                \item $\set{ x \in \mathbb{R} : sin(x) = sign(x)}$
            \end{enumerate}
        \end{enumerate}
    \end{enumerate}

    \item 
    \begin{enumerate}[label=(\alph*), itemsep=10pt]
        \item $\abs{\pset{\pset{\pset{A}}}} = 2^{2^{2^m}}$
        \item $\abs{\pset{A}\times\pset{B}} = 2^{m + n}$
        \item $\abs{\set{X \in \pset{A} : \abs{X} \leq 1}} = m + 1$
        \item $\abs{\set{X \subseteq \pset{A} : \abs{X} \leq 1}} \stackrel{?}{=} m + 1$ 
    \end{enumerate}

    \item
    \begin{enumerate}[label=(\alph*), itemsep=16pt]
        \item insert venn diagrams
        \begin{enumerate}[label=(\roman*), itemsep=10pt]
            \item $B \setminus A$
            \item $(A \setminus B) \cap C$
            \item $(A \setminus B) \cup C$
            \item $(A \cup B) \cap C$
        \end{enumerate}

        \item 
        \begin{enumerate}[label=(\roman*), itemsep=10pt]
            \item $\bigcup_{i \in \N} \bigl[i,i+1\bigr] = \R$
            \item $\bigcap_{i \in \N} \bigl[0,i+1\bigr] = \set{ x \in \R : 0 \le x \le 2 }$
            \item $\bigcap_{a \in \R} \bigl(\set{a} \times [0, 1]\bigr) = \set{ (x,y) : x \in [0,1] \text{ and } y \in [0,1] }$
        \end{enumerate}
    \end{enumerate}
\end{enumerate}
    

\subsection*{Introduction to Mathematical Logic}
\begin{enumerate}
    \item 
    \begin{enumerate}[label=(\alph*), itemsep=16pt]
        \item 
        \begin{itemize}
            \item \underline{Proposition}: 0 is an integer. \\
                \textit{The statement contains the word "is", which is equivalent to the "=" logical operator indicating a truth value; the statement is true because 0 is an integer.}
            \item \underline{Not a Proposition}: To be or not to be? \\
                \textit{The statement indicates no truth value and is posing a philisophical question from Hamlet.}
        \end{itemize}

        \item
        \begin{enumerate}[label=(\roman*), itemsep=10pt]
            \item $P:\;x=0$ and $Q:\;y=0$ \\
            $P\land \neg Q$

            \item $P:$ \textit{determinent is 0} and $Q:$ \textit{matrix is invertible} \\
            $P \implies \neg Q$ 

            \item $P:\;$
        \end{enumerate}
    \end{enumerate}
\end{enumerate}

