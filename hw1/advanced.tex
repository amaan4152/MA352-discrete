\begin{enumerate}
    \item
          \begin{enumerate}[label=(\alph*), itemsep=10pt]
              \item
                    \begin{enumerate}
                        \item
                              Let $a \in A$ and $r \in \Z$. By definition of $A$, $a = 6r + 12 = 3(2r + 4)$. Because $r \in \Z$ and $2r + 4 \in Z$, $a \in B \implies A \subset B$.
                        \item
                              $A \neq B$ holds true if there is an element in $B$ not in $A$. For instance, given that $n \in B$, lets say $n = 3$; then, $s = 1 \in \Z$. However, for $m \in A$, $m$ cannot be 3 because there is no such $r \in \Z$ that satisfies $3 = 6r + 12$.
                        \item
                              Thus, $A$ is a proper subset of $B$ because $A \subset B \text{ and } A \neq B$.
                    \end{enumerate}
              \item False. \\
                    Let $A = \set{1,2,3}$, $B = \set{1,2,3,4,5}$, $C = \set{2,3,4}$, and $D = \set{2,3,4,5,6}$. Then, $A \subseteq B$ and $C \subseteq D$. $A \setminus C = \set{1,4} \text{ and } B \setminus D = \set{1,6} \implies A \setminus C \subsetneq B \setminus D$.

              \item
                    Given that $\bigcup_{i=1}^{\infty} A_i= \bigcap_{i=1}^{\infty} A_i$, then all $A_i$'s must be equal since the the unions and intersections of $A_i$'s are equal.
          \end{enumerate}
\end{enumerate}